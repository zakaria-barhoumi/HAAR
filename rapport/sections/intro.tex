% --- SECTION 1 : INTRODUCTION & ORIGINES ---

\section{Introduction}
\begin{frame}
\frametitle{Introduction} 

\begin{block}{Stéganographie \--- Etymologie}
du grec \emph{steganos}, caché ; et \emph{graphein}, écrire
\end{block}

\begin{block}{Rappel : Stéganographie}
L'art de dissimuler le fait qu'un message existe. Contrairement à la cryptographie qui rend le message illisible, la stéganographie le rend invisible.
\end{block}

\begin{block}{Les vecteurs étudiés aujourd'hui}
Si l'image est le support le plus connu, d'autres médias sont couramment utilisés :
\begin{itemize}
    \item[•] \textbf{Le Texte} : Modification de la structure, des espaces ou de la police.
    \item[•] \textbf{L'Audio} : Exploitation des limites de l'oreille humaine (HAS).
\end{itemize}
\end{block}
\end{frame}

\section{Origines et Applications}
\begin{frame}
\frametitle{Origines et Applications (1/2)}
\begin{block}{Origines}
La stéganographie existe depuis longtemps, bien avant l’invention de l’ordinateur (500 av. J.C.).
\end{block}
% ... (Le reste de vos slides d'intro ici)
\end{frame}