% --- SECTION 2 : STÉGANOGRAPHIE TEXTE ---

\section{Stéganographie dans le Texte}
\begin{frame}
\frametitle{Stéganographie Textuelle}
\framesubtitle{Approches générales}

La stéganographie textuelle est considérée comme l'une des plus difficiles car le texte contient peu d'informations redondantes par rapport à une image ou un son.

\begin{exampleblock}{Trois grandes familles de méthodes}
\begin{itemize}
    \item[1] \textbf{Formatage} : Modification de la forme (police, taille, espacement).
    \item[2] \textbf{Génération sémantique} : Création de phrases artificielles pour cacher un message (Mimic functions).
    \item[3] \textbf{Linguistique} : Utilisation de synonymes ou de structures syntaxiques variables.
\end{itemize}
\end{exampleblock}
\end{frame}

\begin{frame}
\frametitle{Le Chiffre de Bacon (Méthode Bilitère)}
% ... (Suite de vos slides texte)
\end{frame}